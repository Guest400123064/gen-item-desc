\documentclass[12pt]{article}
\usepackage{enumerate}
\usepackage{url}
\usepackage{listings}
\usepackage{upquote,textcomp}
\usepackage{dsfont}
\usepackage{indentfirst}
\usepackage{amsmath}
\usepackage{amsthm}
\usepackage{mathtools}
\usepackage{color}
\usepackage{graphicx}
\usepackage{caption}
\usepackage{subcaption}
\usepackage{amsfonts}
\usepackage{float}
\usepackage{pythonhighlight}


\usepackage{cleveref}
\crefformat{section}{\S#2#1#3} % see manual of cleveref, section 8.2.1
\crefformat{subsection}{\S#2#1#3}
\crefformat{subsubsection}{\S#2#1#3}

\usepackage{booktabs}
\usepackage{dcolumn}
\newcolumntype{d}[1]{D{.}{.}{3}}
\newcommand{\mlc}[1]{\multicolumn{1}{c}{#1}}

\setlength{\parindent}{0cm}
\setlength\parindent{24pt}
\setlength{\parskip}{0.3cm plus4mm minus3mm}

\oddsidemargin = 0.01 in 
\textwidth = 6.5 in
\textheight = 9.8 in
\headsep = -1in

\lstset{
  frame=tb,
  aboveskip=3mm,
  belowskip=3mm,
  showstringspaces=false,
  columns=flexible,
  keepspaces=true,
  basicstyle={\small\ttfamily},
  numbers=none,
  numberstyle=\tiny\color{black},
  keywordstyle=\color{black},
  commentstyle=\color{black},
  stringstyle=\color{black},
  breaklines=true,
  breakatwhitespace=true,
  tabsize=3
}

\DeclareMathOperator*{\argmin}{argmin}

\title{\textbf{Nobody Writes Docstring:} Game Item Description Generation with Pretrained Language Models}
\author{Ning Wan, Hongyu Zhang, Yuxuan Wang}
\date{\today}
\begin{document}

\maketitle

% -------------------------------------------------------------------------------------
\section{\bf Project Description}
  Nobody wants to write docstring. What if we can build a better description generator? 
    Recall the games you played; the item descriptions contains information particular to 
    the game settings or the game-world settings For example, you know the item is from \textit{Dark Soul} or \textit{Don't Starve} 
    simply by reading the text: even when they are not! Therefore, our work is more 
    developer-oriented, particularly convenient for those who want to develop mods for 
    their favorite games and conform to a consistent world setting or style of writing. So, in this project, 
    we would like to revisit Homework Two, the description generation. 
    We want a generator that takes an item title as input and generates a 
    corresponding "gamified" item description. However, instead of building an 
    end-to-end system as we have done in Homework Two, we observe that such a task 
    is two-fold:
    \begin{itemize}
      \item Generate a basic description for a simplified item name that depicts the main properties of this item. 
        For instance, a sword is made of metal, a pointy weapon.
      \item Render the utterance to sound like a game item when we condition it on the game category and full 
        item name like \textit{King Arthur's Sword}.
    \end{itemize}
    To sum, we want to formulate this task as a combination of description generation and conditioned style transfer problem.
 
% =================================================================================================
\section{Related Works}
  Our project leverages the recent advancement of large \textbf{Pre-trained Language Models} (PLM), e.g., 
    GPT-3, T5, and BART \cite{gpt-3, t5, bart}. Further, it is shown that such PMLs can serve as knowledge bases, 
    especially salient for simple factual knowledge \cite{comet, kblm}. This inspires us of using large PLMs as 
    base description generator.

  Our formulation of game item description generation resembles the most to the \textbf{Text Style Transfer} (TST) 
    task with some variation. 
    TST aims to control certain attributes in the generated text, such as politeness, emotion, humor, 
    and many others \cite{tst_survey}. It involves a source style and a target style. In our case, the source 
    style is the simple item description, where as the target style is the gamified description. Specifically, 
    given the base description, we can construct a parallel dataset where standard sequence-to-sequence models 
    can be applied \cite{s2s_style}.

    However, we need to keep in mind that the renderer functions not only as a style transferer but also as a 
      domain-specific knowledge base. For example, to generate fantasy-game-related item descriptions, a renderer 
      needs to have prior knowledge about terms like Mana, Goblin, or Holy Grail. And we do not expect an out-of-the-box 
      model has such knowledge pre-encoded \cite{domain_adaptation}. Therefore, the specific methods to solve this problem are under discussion.

% =================================================================================================
\section{Methods}
  \subsection{Proposed Architecture}
    According to our observation, we can decompose the game-item description 
      generation as a two-phase process: base form description generation and rendering.
      The two models are collated together to form a game item descriptor.

    \subsubsection{Base Generator}
    First of all, almost every game item has a "base form." Such base forms can be weapons, 
      garments, or food. The base form of a game item reveals its essential properties. 
      For instance, \textit{Excalibur} is still a sword, a pointy weapon (possibly) made of a 
      piece of metal. Therefore, intuitively, to generate a description for a game item, 
      we need a particular mechanism to help create a piece of text for its base form in 
      the first place. Such a machine needs to possess common sense knowledge about the 
      properties of various base forms, which naturally leads to large PLMs or models 
      trained upon knowledge bases. Formally, a generator is a function with the signature,
      \begin{python}

          def generator(base_form: str) -> str:
              """The generator model API. It should take a single 
                  item like 'sword' and generate some common-sense-based 
                  description for that item, e.g. 'a pointy piece of metal'.
                  The <model> here is an abstraction, it can be a large PLM 
                  like GPT-3 or specialized knowledge-base model like COMET-BART"""

              prompt = make_prompt_gen(base_form)
              base_desc = model_gen(prompt)
              return post_proc_gen(base_desc)
      \end{python}

    \subsubsection{Game Renderer}
    Next, we believe that two extra pieces of information contribute to the "game flavor" 
      of the description. Firstly, where do these items appear? For instance, a cup in a 
      dungeon may associate with adjectives like "filthy" or "rusty," while a mug in a 
      kitchen can be "clean" or "cute." Secondly, the full name of that item, e.g., 
      \textit{King Arthur Sword} instead of \textit{Sword}. We know that this particular weapon is from 
      \textit{King Arthur's Legend}. So the corresponding descriptions should relate to terms like 
      \textit{Avalon}, \textit{Knights of the Round Table}, or \textit{Holy Grail}. We can see that it also requires 
      the renderer to have another set of common sense knowledge. However, such knowledge 
      is more game-specific or world-specific: it is not likely to have a machine gun in 
      greek mythology. In other words, the renderer serves as an (inverse) adapter, and 
      the engineering advantages brought by the disentanglement are apparent, i.e., we do 
      not need to re-train the often-humongous base generator when switching to a different 
      set of settings or game. 
      \begin{python}

        def renderer(setting: str, full_name: str, base_desc: str) -> str:
            """The renderer model API. It rephrase/diversify the item 
                description conditioning on the <setting>, <full_name>, 
                and pre-computed <base_desc>"""

            prompt = make_prompt_ren(setting, full_name, base_desc)
            game_desc = model_ren(prompt)
            return post_proc_ren(game_desc)
      \end{python}

    \subsubsection{Item Base Form Extractor}
    Another small but important piece of building block is the item base form extractor. 
      Ideally, a user of our proposed system only needs to specify the full name of a game 
      item, but a base generator only takes the \textbf{base form} as its input. Thus, we 
      need a dedicated algorithm to extract/convert the game item to its base form. We 
      currently explored two approaches. The first method is based on KeyBERT, which 
      tries to find a subset of the tokens that best preserves the semantic information, measured 
      by \textbf{cosine-similarity} between full name contextualized embedding and those of the sub-tokens'. However, the base form may not necessarily contained in the full name. So, another
      method we proposed is to, again, rely on large PLMs and text-generation. We experimented 
      with generation using COMET-BART via \textit{MadeOf} relation, and prompting-by-example with 
      GPT-3. The GPT-3 model gives the best performance.
  
  \subsection{Model Development Procedure}
    We introduce both planned development procedure as well as 
      preliminary experiments in this section.

    \subsubsection{Generator Construction}
      \paragraph{Preliminary Experiments} As discussed above, the base description generation relies on a 
        large PLM that can serve as a knowledge base. Therefore, we develop our baseline model based 
        on \textbf{GPT-3} via \textbf{few-shots prompting}. We construct the prompt by concatenating 
        ten \lstinline{"item: <ITEM>, description: <DESC>"} pairs. An example is shown below,
        \begin{python}

              prompt = f"""
              item: Sunglasses, description: All of our sung...
              item: Scarf, description: Feature -Great quality...
                                  ...
              item: Shoe, description: Kick back and relax...
              item: {NEW_ITEM}, description:
              """
        \end{python}
        The item descriptions we used is acquired from the Amazon Review Dataset, which is discussed 
          in \cref{section:data}. This simple method works surprisingly well, which is discussed in
          \cref{section:pre_results}.
      
      \paragraph{Planned Procedures} Although the preliminary result is satisfactory, prompting \lstinline{davinci} model
        is very costly. Thus, our next step is to collect sufficient amount of examples, roughly a hundred, and fine-tune a 
        smaller model such as \lstinline{ada} and \lstinline{babadge}. Further, we have not yet fully experimented various 
        crucial sampling hyperparameter such as the temperature and penalties. 

    \subsubsection{Renderer Construction}
      \paragraph{Preliminary Experiments} Again, we use \textbf{GPT-3 prompting} for the proof-of-concept purpose. 
        An example prompt is shown below,
        \begin{python}
          
              prompt = f"""
              item: Wood table category: Tavern base_desc: The table is made of wood ... desc: The table is old and dusty... ===
                                  ...
              item: {FULL_NAME} category: {SETTING} base_desc: {BASE_DESC} desc:
              """
        \end{python}
        The base description is generated from pre-constructed base-desc-generator, and target description 
          is taken from the LIGHT dataset, discussed in \cref{section:data}.

      \paragraph{Planned Procedures} Since the prompting performance is under expectation, we plan to 
        perform error analysis first and discuss what information is not captured by the model, or, 
        if we need to fine-tune the model on full LIGHT dataset or other augmented datasets.

% =================================================================================================
\section{Data}
\label{section:data}
  \subsection{Amazon Review (Meta) Data} 
    This dataset contains product reviews and metadata from Amazon, including 142.8 million reviews 
      spanning from May 1996 to July 2014. This dataset includes reviews (ratings, text, helpfulness votes), 
      product metadata (descriptions, category information, price, brand, and image features), and 
      links (also viewed/also bought graphs) \footnote{https://jmcauley.ucsd.edu/data/amazon/}. For this project, \textbf{we only use the metadata} which 
      contains \textbf{item names} and corresponding \textbf{short descriptions}. But, we noticed that 
      the dataset is noisy (for our purpose). For instance, it may include advertising 
      text that is depicting the properties of the (base) item. Thus, we manually select a tiny subset of 
      descriptions that are more relevant. Further, we plan to augment the dataset with Wikipedia descriptions.

  \subsection{LIGHT} We briefly discuss the LIGHT dataset since we frequently work with this dataset throughout
    this course. At this stage, we only use the objects dataset. For each one of the game items, there are 
    \lstinline{'name'}, \lstinline{'in_room_ids'}, \lstinline{'base_form'}, and \lstinline{'descriptions'}. 
    To formulate a style transfer task, we first apply 
    the pre-constructed base generator to \lstinline{'base_form'} to make a pseudo-parallel dataset. 
    Next, for the rooms an item exists in, we use the \lstinline{'category'} of the rooms to control the 
    style transfer (use as \lstinline{'setting'}). Finally, we have the required information for the renderer.

% =================================================================================================
\section{Evaluation Plan}
The most important component of our system is the renderer. We primarily focus on evaluating lexical overlapping and 
  semantic preservation. We use the \textbf{BLEU Score} and \textbf{ROUGE Score} 
  to measure the lexical overlap. The reason for including the ROUGE score is to assess the recalled properties of 
  gamified description. To evaluate the degree of semantic preservation, we plan to use SentenceBERT to compute the 
  semantic similarity between the generated text and the ground truth. The perfect result would be the two pieces of 
  texts being almost paraphrases of each other. We discard \textbf{Perplexity} evaluation because the game item descriptions may naturally be very complex 
  because the "uncommon words," i.e., game-specific words, are very common. 

Although our main focus is to evaluate the renderer, we also understand that the base descriptions are 
  strong baseline. Thus, we will compare the generation quality against the "raw descriptions."

% =================================================================================================
\section{Preliminary Results}
\label{section:pre_results}
  \subsection{Base Generator}
    At the point of writing this milestone report, we have constructed baseline models for both of 
      the generator and renderer based on GPT-3 prompting. The base generator achieves satisfactory 
      generation quality, examples are shown below.
      \begin{itemize}
        \item \textbf{Knife}: \textit{The knife is a tool that has been used by humans for centuries. It is a versatile tool that can be used for a variety of purposes, 
          including cutting, slicing, and carving. A knife is typically composed of a blade and a handle. 
          The blade is the sharpened edge of the knife that is used for cutting, while the handle is used 
          for grip and control. There are a variety of different types of knives, including pocket knives, 
          kitchen knives, and hunting knives.} 
        \item \textbf{Necklace}: \textit{This beautiful cowrie shell necklace can fit any neck size! 
          It looks great and is strung on a comfortable and strong nylon cord, which is also waterproof. 
          Imported from Africa.}
      \end{itemize}

    One interesting phenomenon is that when we plugin \textit{King Arthur Sword}, the generator 
      gives \textit{"The King Arthur sword is a replica of the sword that was used by King Arthur 
      in the famous legend. The sword is made of high quality stainless steel and has a polished 
      finish. It is also very sharp and can be used for display or cosplay purposes."}
      It shows that, without rendering, the base descriptor produce items 
      that are derived from day-today life. Even the king's sword only serves as a costume. This 
      proves the necessity of a renderer.

    Further, we experiment with COMET-ATOMIC-BART, but the out-of-the-box performance is not as satisfactory. Most 
      generated descriptions are too short, i.e., not informative enough. We hypothesis that a more informative 
      description requires a combination of relations more than just \lstinline{MadeOf}, which needs 
      heavy engineering, beyond the scope of prototyping.

  \subsection{Game Renderer}
    \begin{python}
          
      >>> baseline("king arthur sword", "magical realm", get_description("sword"))
      'The sword is old and rusted, but still sharp. The blade is engraved with the words "Excalibur".'
      >>> baseline("teeth of troll", "dungeon", get_description("teeth"))
      'The teeth are sharp and white, and they look like they could belong to a troll. They are large and sharp, and would be dangerous if used as a weapon.'
      >>> baseline("enchanted bone", "Graveyard", get_description("bone"))
      'The bone is old and dry. It is long and hard, and can be used as a weapon. The bones are old and crumbling, showing a storied past.'
    \end{python}

    We randomly selected ten objects from the LIGHT dataset to construct the few-shots prompt. Then, we extracted the object name, 
      the category (style) of the room that it was in, and the original object description for each object. The prompt is the 
      object's full name, the room category, and the base description given the base form of the object name using the base 
      generator we constructed above. The completion is the intended game object description.

\section{Attribution}
  \begin{itemize}
    \item \textbf{Ning Wan}, constructs the base description generator baseline as well as item base-form extractor; will focus on improving the quality of the sub-models. 
    \item \textbf{Hongyu Zhang}, constructs the renderer baseline; will focus on further improving the rendering quality.
    \item \textbf{Yuxuan Wang} initiates idea and manages the progress of the project; also responsible for the report writing.  
  \end{itemize}

% -------------------------------------------------------------------------------------
\bibliography{milestone2.bib}

\bibliographystyle{acm}

\end{document}
